\documentclass[12pt,fleqn]{article}
\usepackage{mathtools}
\usepackage{amsmath}


\begin{document}
% \setlength{\mathindent}{0pt}
\section{Distributions}
\subsection{how to tell between the distributions?}

generallly counting processes are discrete (how many coin flips)

amount of time, speed, height, continuous/interval processes are continuous. measured using an interval (you dont go from 500kph to 501 kph without passing through 500.5 kph)



\subsection{Geometric}
Discrete

"how many flips until first success"

$E[k, p] = \dfrac{1-p}{p}$ or $\dfrac{1}{p}$

$\sigma = \dfrac{1-p}{p^2}$

\subsubsection{PMF}
$Geo(k, p) = (1-p)^{k-1}p$

\subsection{Bernoulli}
\subsection{Binomial}
Discrete

\begin{align*}
	E[x] = np\\
	Var(x) = np(1-p)
\end{align*}

\subsubsection{PMF}
$bin(x=k,n,p) = {N\choose k} p^k(1-p)^n-k$

assumes resilts are independent and identically distributed


\subsection{Negative Binomial}
\subsection{Poisson}
Discrete

mean arrival rate ($\mu$) is usually given

$\sigma = \sqrt{\mu}$

\subsubsection{PDF}
$p_x(k;\mu) = p(X=k; \mu) = \dfrac{e^{-\mu}\mu^k}{k!}$, for $k = 0,1,2...$
\subsection{Uniform}

\begin{align*}
	f(x) &= \dfrac{1}{(b-a)}, \text{for $a \le x \le b$}\\
	f(x) &= 0, \text{for any single $x$}\\
	E[x] &= \dfrac{(b-a)}{2}\\
	Var(x) &=\dfrac{(b-a)^2}{12}
\end{align*}

\subsection{Normal}

median is $\mu$

\begin{align*}
	\mu_{discrete} &= \sum_{x \in D} x p(x)\\
	E[x]_{discrete} &= \sum_{x \in D} h(x) p(x)\\
	\mu_{continuous} &= \int_{-\infty}^{\infty} x f(x) dx\\
	E[x]_{continuous} &= \int_{-\infty}^{\infty} h(x) f(x) dx
\end{align*}

\subsubsection{PDF}
$f(x) = \dfrac{1}{\sigma\sqrt{2\pi}}e^{(\dfrac{-(x-\mu)^2}{2\sigma^2})}$
\subsubsection{CDF}
$F(x) = $

\subsubsection{Standardizing}
a random variable following the standard normal distribution $\Phi$ has $\mu = 0$ and $\sigma = 1$ and is often represented by the letter $Z$.

This can be converted to/from a non-standard normal using these formulas:

$Z = \dfrac{X-\mu}{\sigma}$


$X = (Z \sigma) + \mu$


\subsection{Exponential}


$
f(x)=\begin{cases}
\lambda e^{-\lambda x}, & \text{for $x\ge0$}.\\
0, & \text{for $x<0$}.
\end{cases}
$

$ F(x)=1-e^{-\lambda x} $
\subsection{Gamma}


\subsection{Sample Means}

$\sigma$ - Population standard deviation

$\mu$ - Population mean

$\overline{X}$ - Sample mean

$T_0$ - Sample total

\begin{align*}
T_0 &= n\mu\\
Var(T_0) &= \sqrt{n\sigma}
\end{align*}

\end{document}