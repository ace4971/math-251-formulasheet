\documentclass[12pt,fleqn]{article}
\usepackage{mathtools}
\usepackage{amsmath}


\begin{document}
% \setlength{\mathindent}{0pt}

\section{how to tell between the distributions?}

generallly counting processes are discrete (how many coin flips)

amount of time, speed, height, continuous/interval processes are continuous. measured using an interval (you dont go from 500kph to 501 kph without passing through 500.5 kph)



\section{Geometric}
\section{Bernoulli}
\section{Binomial}
\section{Negative Binomial}
\section{Poisson}
Discrete

mean arrival rate ($\mu$) is usually given

$\sigma = \sqrt{\mu}$

\subsection{PDF}
$p_x(k;\mu) = p(X=k; \mu) = \dfrac{e^{-\mu}\mu^k}{k!}$, for $k = 0,1,2...$
\section{Uniform}

\begin{align*}
	f(x) &= \dfrac{1}{(b-a)}, \text{for $a \le x \le b$}\\
	f(x) &= 0, \text{for any single $x$}\\
	E[x] &= \dfrac{(b-a)}{2}\\
	Var(x) &=\dfrac{(b-a)^2}{12}
\end{align*}

\section{Normal}

median is $\mu$

$\mu_{discrete} = \sum_{x \in D} x p(x)$

$E[x]_{discrete} = \sum_{x \in D} h(x) p(x)$

$\mu_{continuous} = \int_{-\infty}^{\infty} x f(x) dx$

$E[x]_{continuous} = \int_{-\infty}^{\infty} h(x) f(x) dx$

\subsection{PDF}
$f(x) = \dfrac{1}{\sigma\sqrt{2\pi}}e^{(\dfrac{-(x-\mu)^2}{2\sigma^2})}$
\subsection{CDF}
$F(x) = $

\subsection{Standardizing}
a random variable following the standard normal distribution $\Phi$ has $\mu = 0$ and $\sigma = 1$ and is often represented by the letter $Z$.

This can be converted to/from a non-standard normal using these formulas:

$Z = \dfrac{X-\mu}{\sigma}$


$X = (Z \sigma) + \mu$


\section{Exponential}


$
f(x)=\begin{cases}
\lambda e^{-\lambda x}, & \text{for $x\ge0$}.\\
0, & \text{for $x<0$}.
\end{cases}
$

$ F(x)=1-e^{-\lambda x} $
\section{Gamma}


\section{Sample Means}

$\sigma$ - Population standard deviation

$\mu$ - Population mean

$\overline{X}$ - Sample mean

$T_0$ - Sample total


$T_0 = n\mu$
$Var(T_0) = \sqrt{n\sigma} $


\end{document}